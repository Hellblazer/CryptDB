\appendix

We describe in more detail $\fu(Q)$ (defined in \S\ref{s:def}), for any $Q$
an SQL query. A relation is an expression of the form \texttt{\{x OP y\}}, where \texttt{OP} can be any
operation (such as $=, >, <, \ge, \le$, \texttt{ILIKE}) that can be performed
between columns and/or constants. $x$ and $y$ are either names of columns in
a database or the wildcard $*$ (meaning any constant value). An instantiation of a relation is an expression where each column operand is
replaced with the index of an item in the column, and $*$ is
replaced with some value. For example, an instantiation of \texttt{\{table1.c1 =
const\}} is \texttt{\{item i of table1.c1 = x1c5a21\}}. The evaluation of an
instantiation is a boolean value: true if the instantiation of the relation is
true and false otherwise.

Then, $\fu(Q)$ is the set of all relations for which there is an instantiation
whose evaluation the server needs to know in order to perform SQL processing
given $Q$.

For example, let \texttt{Q: SELECT MAX(table1.c1) FROM table1, table2} \texttt{
WHERE table1.c2 = table2.c1 AND table2.c2 = x1c5a21}. Then, $\fu(Q) = \{$
\texttt{\{table1.c1 > table1.c1\}} (for \texttt{MAX}), \texttt{\{table1.c2 =
table2.c1\}}, \texttt{\{table2.c2 = *\}} $\}$. For joins, if we have a
predicate of the form \texttt{c1 = c2}, this introduces \{\texttt{c1 = c2}\},
\{\texttt{c1 = *}\} and \{\texttt{c2 = *}\} in $\fu()$. Inserts and aggregates
do not increase $\fu()$. Updates by incrementation do not increase $\fu()$ either; Updates by setting a value such as \texttt{SET c1 = x1c5a21} add $c1 = *$ to $\fu{}$; Deletes add relation corresponding to their filters as discussed above.