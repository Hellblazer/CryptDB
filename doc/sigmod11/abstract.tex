\begin{abstract}

  \emph{CryptDB} is a DBMS that provides provable and practical
  privacy in the face of a compromised database server or curious
  database administrators. CryptDB works by executing SQL queries over
  encrypted data.  At its core are three novel ideas: an
  \textit{SQL-aware encryption strategy} that maps SQL operations
  to encryption schemes, {\em adjustable query-based encryption}
  which allows CryptDB to adjust the encryption level of each
  data item based on user queries, and \textit{onion encryption} to
  efficiently change data encryption levels.
  \name{} only empowers the server to execute queries that the users
  requested, and achieves maximum privacy given the mix of queries
  issued by the users. The database server fully evaluates queries
  on encrypted data and sends the result back
  to the client for final decryption; client machines do not perform
  any query processing and client-side applications run unchanged. Our
  evaluation shows that CryptDB has modest overhead: on the TPC-C
  benchmark on Postgres, CryptDB reduces throughput by $\tput$ compared
  to regular Postgres. Importantly, CryptDB does not change the
  innards of existing DBMSs: we realized the implementation of CryptDB
  using client-side query rewriting/encrypting, user-defined
  functions, and server-side tables for public key information. As
  such, CryptDB is portable; porting CryptDB to MySQL required
  changing $86$ lines of code, mostly at the connectivity layer.

\end{abstract}
